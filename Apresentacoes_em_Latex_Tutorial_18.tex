\documentclass{beamer}\usepackage[]{graphicx}\usepackage[]{color}
%% maxwidth is the original width if it is less than linewidth
%% otherwise use linewidth (to make sure the graphics do not exceed the margin)
\makeatletter
\def\maxwidth{ %
  \ifdim\Gin@nat@width>\linewidth
    \linewidth
  \else
    \Gin@nat@width
  \fi
}
\makeatother

\definecolor{fgcolor}{rgb}{0.345, 0.345, 0.345}
\newcommand{\hlnum}[1]{\textcolor[rgb]{0.686,0.059,0.569}{#1}}%
\newcommand{\hlstr}[1]{\textcolor[rgb]{0.192,0.494,0.8}{#1}}%
\newcommand{\hlcom}[1]{\textcolor[rgb]{0.678,0.584,0.686}{\textit{#1}}}%
\newcommand{\hlopt}[1]{\textcolor[rgb]{0,0,0}{#1}}%
\newcommand{\hlstd}[1]{\textcolor[rgb]{0.345,0.345,0.345}{#1}}%
\newcommand{\hlkwa}[1]{\textcolor[rgb]{0.161,0.373,0.58}{\textbf{#1}}}%
\newcommand{\hlkwb}[1]{\textcolor[rgb]{0.69,0.353,0.396}{#1}}%
\newcommand{\hlkwc}[1]{\textcolor[rgb]{0.333,0.667,0.333}{#1}}%
\newcommand{\hlkwd}[1]{\textcolor[rgb]{0.737,0.353,0.396}{\textbf{#1}}}%
\let\hlipl\hlkwb

\usepackage{framed}
\makeatletter
\newenvironment{kframe}{%
 \def\at@end@of@kframe{}%
 \ifinner\ifhmode%
  \def\at@end@of@kframe{\end{minipage}}%
  \begin{minipage}{\columnwidth}%
 \fi\fi%
 \def\FrameCommand##1{\hskip\@totalleftmargin \hskip-\fboxsep
 \colorbox{shadecolor}{##1}\hskip-\fboxsep
     % There is no \\@totalrightmargin, so:
     \hskip-\linewidth \hskip-\@totalleftmargin \hskip\columnwidth}%
 \MakeFramed {\advance\hsize-\width
   \@totalleftmargin\z@ \linewidth\hsize
   \@setminipage}}%
 {\par\unskip\endMakeFramed%
 \at@end@of@kframe}
\makeatother

\definecolor{shadecolor}{rgb}{.97, .97, .97}
\definecolor{messagecolor}{rgb}{0, 0, 0}
\definecolor{warningcolor}{rgb}{1, 0, 1}
\definecolor{errorcolor}{rgb}{1, 0, 0}
\newenvironment{knitrout}{}{} % an empty environment to be redefined in TeX

\usepackage{alltt}
\usepackage{graphicx}
\IfFileExists{upquote.sty}{\usepackage{upquote}}{}
\begin{document} 
\begin{frame}
\frametitle{Escuela de Invierno de Métodos (EIM)} 
\vspace{0.5cm}
Universidad Católica de Uruguay (UCU)
\vspace{0.5cm}

Montevideo, Uruguay

\vspace{0.25cm}
15-26, Julio de 2019
\vspace{0.25cm}
\end{frame}



\begin{frame}
\frametitle{Escuela de Invierno de Métodos (EIM)} 
\framesubtitle{Introducción a la Estadística Inferencial - del 15 al 19 de julio - de las 8:30 a las 12:30}
\vspace{0.5cm}

Descripción

\vspace{0.25cm}

El curso familiariza a los estudiantes con el razonamiento estadístico introduciendo los conceptos y herramientas básicos para el análisis de problemas existentes en la investigación en ciencias sociales. Se espera que los estudiantes adquieran los conocimientos necesarios de probabilidad y estadística inferencial para un posterior curso de introducción a la econometría. El curso hace fuerte énfasis en los conceptos teóricos de la estadística inferencial así como en el trabajo aplicado con datos.

\vspace{0.25cm}

Entre los contenidos se abordarán los siguientes: poblaciones, muestras y validez, nociones básicas de probabilidad, aproximación normal a los datos, muestreo, inferencia estadística y tests.
\vspace{0.25cm}
\end{frame}

\begin{frame}
\frametitle{Escuela de Invierno de Métodos (EIM)} 
\framesubtitle{Matemáticas para las Ciencias Sociales - del 15 al 19 de julio - de las 18:00 a las 21:00}
\vspace{0.5cm}

Descripción 

\vspace{0.25cm}

El curso se impartirá en dos etapas. En la primera etapa se revisarán algunos conceptos básicos vinculados a la teoría de conjuntos, conjuntos numéricos, operaciones algebraicas, funciones elementales y teoría de matrices. En la segunda parte se introducirá el concepto de límite, derivada e integración de funciones reales y elementos básicos de la teoría de probabilidad.

\vspace{0.25cm}
\end{frame}

\begin{frame}
\frametitle{Escuela de Invierno de Métodos (EIM)} 
\framesubtitle{Introducción a regresión lineal y análisis multivariado - del 22 al 26 de julio - de las 8:30 a las 12:30}
\vspace{0.5cm}

Descripción

\vspace{0.25cm}

Este es un curso introductorio en técnicas cuantitativas para el análisis social, orientado a preparar técnicos que logren aplicar inteligentemente las herramientas estadísticas básicas en este campo. El curso combinará clases sobre teoría estadística básica y practica en la modelización estadística. En particular se revisarán medidas de asociación, construcción de índices y factores, y modelos básicos de regresión lineal. Se alternarán exposiciones teóricas con sesiones prácticas de laboratorio.

\vspace{0.25cm}
\end{frame}

\begin{frame}
\frametitle{} 

\vspace{0.5cm}

\includegraphics[width=\linewidth]{logo_ucu.png}

\vspace{0.25cm}
\end{frame}

\end{document}


