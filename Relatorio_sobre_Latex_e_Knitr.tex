\documentclass{article}\usepackage[]{graphicx}\usepackage[]{color}
%% maxwidth is the original width if it is less than linewidth
%% otherwise use linewidth (to make sure the graphics do not exceed the margin)
\makeatletter
\def\maxwidth{ %
  \ifdim\Gin@nat@width>\linewidth
    \linewidth
  \else
    \Gin@nat@width
  \fi
}
\makeatother

\definecolor{fgcolor}{rgb}{0.345, 0.345, 0.345}
\newcommand{\hlnum}[1]{\textcolor[rgb]{0.686,0.059,0.569}{#1}}%
\newcommand{\hlstr}[1]{\textcolor[rgb]{0.192,0.494,0.8}{#1}}%
\newcommand{\hlcom}[1]{\textcolor[rgb]{0.678,0.584,0.686}{\textit{#1}}}%
\newcommand{\hlopt}[1]{\textcolor[rgb]{0,0,0}{#1}}%
\newcommand{\hlstd}[1]{\textcolor[rgb]{0.345,0.345,0.345}{#1}}%
\newcommand{\hlkwa}[1]{\textcolor[rgb]{0.161,0.373,0.58}{\textbf{#1}}}%
\newcommand{\hlkwb}[1]{\textcolor[rgb]{0.69,0.353,0.396}{#1}}%
\newcommand{\hlkwc}[1]{\textcolor[rgb]{0.333,0.667,0.333}{#1}}%
\newcommand{\hlkwd}[1]{\textcolor[rgb]{0.737,0.353,0.396}{\textbf{#1}}}%
\let\hlipl\hlkwb

\usepackage{framed}
\makeatletter
\newenvironment{kframe}{%
 \def\at@end@of@kframe{}%
 \ifinner\ifhmode%
  \def\at@end@of@kframe{\end{minipage}}%
  \begin{minipage}{\columnwidth}%
 \fi\fi%
 \def\FrameCommand##1{\hskip\@totalleftmargin \hskip-\fboxsep
 \colorbox{shadecolor}{##1}\hskip-\fboxsep
     % There is no \\@totalrightmargin, so:
     \hskip-\linewidth \hskip-\@totalleftmargin \hskip\columnwidth}%
 \MakeFramed {\advance\hsize-\width
   \@totalleftmargin\z@ \linewidth\hsize
   \@setminipage}}%
 {\par\unskip\endMakeFramed%
 \at@end@of@kframe}
\makeatother

\definecolor{shadecolor}{rgb}{.97, .97, .97}
\definecolor{messagecolor}{rgb}{0, 0, 0}
\definecolor{warningcolor}{rgb}{1, 0, 1}
\definecolor{errorcolor}{rgb}{1, 0, 0}
\newenvironment{knitrout}{}{} % an empty environment to be redefined in TeX

\usepackage{alltt}

\usepackage[utf8]{inputenc}
\IfFileExists{upquote.sty}{\usepackage{upquote}}{}
\begin{document}

\title{Relatório sobre Latex}
\date{May 31st, 2019}
\author{Akira Medeiros}

\maketitle 

Relatório sobre Latex e Knitr

\vspace{0.5cm}

\textbf{Universidade Catolica do Uruguay}

\textit{Escuela de Invierno de Métodos da Universidade Católica de Uruguay}

\underline{15 a 26 de julho de 2019}

\vspace{0.5cm}

Lista de cursos
\begin{itemize} 
\item Introducción a la Estadística Inferencial 
\item Introducción a regresión lineal y análisis multivariado 
\end{itemize}


\begin{enumerate} 
\item Introducción a la Estadística Inferencial 
\item Introducción a regresión lineal y análisis multivariado
\end{enumerate}

\section{Lista de Cursos}
\subsection{Semana 1}
\subsubsection{Introducción a la Estadística Inferencial}
\subsection{Semana 2}
\subsubsection{Introducción a regresión lineal y análisis multivariado}

Títulos e Secões não numerados:
\section*{Lista de Cursos}
\subsection*{Semana 1}
\subsubsection*{Introducción a la Estadística Inferencial}

Equações em Latex

$$\alpha^2 + \beta^2 = \chi^2$$

$$\frac{\sqrt{1}}{2} * \frac{a}{2b} = \frac{a}{4b}$$ 

$$\sum_0^{10} x = ...$$ 

Nós temos que combater os cavaleiros do apocalispe.

\vspace{0.5cm}
Equação de Pitágoras
\vspace{0.5cm}
$$a^2 = b^2 + c^2$$

\end{document}

